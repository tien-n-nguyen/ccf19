\subsection{Significance of This Proposed Project: NSF Merit Criteria}
\label{significance-section}

The results of this project will advance the state-of-the-art
knowledge and understanding in API usability and maintenance. They are
transformative and directly improve software productivity and quality.

\noindent
{\bf Technical Merits} are addressed throughout this proposal are as
follows:

\begin{itemize}

  \item {\bf Advance the state-of-the-art research, knowledge and
    understanding}. Via our empirical studies in {\em Task 1}, the
    results of this research will advance the state-of-the-art
    knowledge on the nature and characteristics of {\em the
      correspodence beteween texts and code} and how they help in
    automatic API code generation.
%
{\em Our research tasks 2 and 3} will fundamentally advance the body
of knowledge and understanding in theoretical foundations, and
practical techniques in graph-based statistical translation models and
tools to support API code generation to assist developers in
programming.

%support developers in programming and use more effective APIs.


  \item {\bf Scientific foundation and creative/original
    elements}. This project will provide a scientific foundation
    (novel concepts, representation, theories, algorithms, techniques,
    and tools) to (1) mine large-scale data to build parallel corpus
    of texts and code; (2) to advance statistical translation with
    graph synthesis, which is novel to support source code than
    state-of-the-art translation models in NLP; (3) to
    advance the techniques in dealing with incomplete code in
    discussion forums (e.g. StackOverflow).

%for capturing the knowledge on API usages and miuses and API-related
%vulnerabilities, and leverage them to guide the patching process and
%suggesting fixes on other related systems. We will explore approaches
%to characterize, represent, and recognize API misuses and their
%relations to security vulnerabilities.

%Tien

To achieve our goal, we must overcome several key challenges. For task
1 of building a large-scale parrallel corpus, we must find an
automatic method to mine such a large-scale dataset with little effort
from human in manual annotation of corresponding texts and API usages.
Moreover, to mine online resources (e.g., StackOverflow) and
open-source projects, we must handle the missing information in
incomplete code, e.g., the information on the fully qualified names of
APIs and libraries is often implicit and missing. For task 2, we need
(1) an effective representation for API usage code and corresponding
statistical translation models suitable for program semantics; (2) to
make use of large amounts of data; (3) to flexibly support
several languages for any application domains; (4) to effectively
represent program properties in any domain. For task 3, the usability
for our approaches relies on how we present the resulting API usages
to developers in different enviroments. We plan to conduct user
studies to evaluate our tools' performance and usability. Other
challenges will be presented later.

%to create API-Misuse Classification and a taxonomy for API misuses,
%and a framework to assess the capabilities of detectors, we have to
%face the challenges in {\em large-scale code mining} because if
%lacking of examples, we would not be able to observe sufficiently
%large data to classify them into categories of API misuses causing
%security vulnerabilities. Fortunately, we have been using our
%large-scale mining infrastructure, Boa~\cite{boa-icse13} to cover the
%large space of API~misuses.

%
%For task 2, a benchmark engine to store and run the API-misuse
%detectors in a scalable manner for a large number of misuses requires
%an effective and efficient infrastructure. None of dataset or
%benchmark of API misuses and API-related software vulnerabilities
%exists. The benchmark engine has to be as scalable, automated,
%and efficient.
%
%For task 3, we  must answer the following questions: 1)
%what representations are needed to capture API-related vulnerable
%code; what are the boundary of an API usage; what an efficiently
%mining algorithm for the patterns of API usages is, should it be mined
%on the same projects as the ones for vulnerability detection? 3)
%should the mining be specific in certain projects or application
%domains?, 4) should the API-related vulnerable code detection
%algorithm be separated from the pattern detection one; how would we
%provide a ranking for candidates; how would we handle alternative
%usages and semantically correct deviations; how wouldd we
%overcome challenges listed for existing techniques in task 2.

\end{itemize}

\noindent {\bf Broad Impact criteria} are addressed throughout the
proposal as follows:

\begin{itemize}

  \item {\bf Transformative and benefits to society}. Our 
    results will be transformative and directly benefit to our
    society.  They will lead to increasing developers' productivity
    and API usability. Our validation involves students and
    professionals, promoting teaching, training, and learning of APIs
    and libraries.

%report

  \item {\bf Foster other research activities}. Our results will
    foster research activities in related fields such as NLP and
    software maintenance. This project will produce theoretical
    concepts and techniques that are novel even in NLP, e.g.,
    graph-based machine translation models. The collected parallel
    corpus will be useful for software maintenance research, e.g.,
    code search and text-code linking. This project will also advance
    the state-of-the-art research in large-scale program analysis with
    statistical models.

%The representation for software security vulnerabili will be useful in
%research on software security, malware detection, vulnerability
%reports, and automatic security patching.


  \item {\bf Education, dissemination, and broader participation}.
  The research will enhance the infrastructure for teaching and
  research by providing tools and data sets for use by students and
  practitioners, and for enhancement by other researchers. We will
  provide related learning modules for educators as well. It will
  contribute novel teaching modules to our curriculum. Details will be
  presented in Section 6.

%Details are in Section 4.

\end{itemize}
