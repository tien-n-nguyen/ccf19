\section{Task 1. Scientific foundation for the representation for feature-interaction and feature-interaction detection}
\label{task1-section}

\subsection{The representation for feature-interaction}
In a configurable system, a \textit{feature} is an optional
functionality in addition to a special feature - \textit{the core
feature} (i.e., \textit{the base variant}) which is enabled in 
every variant of the system. In C/C++ programs, features are 
implemented as statically conditional statements\footnote{For 
the purpose of localization, we assume that statement is the unit 
of program that need to be considered in the fault localization 
problem.} guarded by constraints over feature symbols (e.g., 
\texttt{\#if} macro). In the running example, feature \texttt{LOCKDEP}
contains the set of statements from line 20 to 27.

\begin{figure}[!h]
\begin{center}
\lstinputlisting[captionpos=b, numbers=left, stepnumber=1, frame=single]{example_2.c}
\caption{A Configuration-dependent Bug in Linux Kernel}
\label{example_bug}
\end{center}
\end{figure}

%Feature
\begin{Definition}{({\bf Feature}).}
A feature is implemented as a set of optional statements that are 
guarded by constraints over feature symbols. Given a configurable 
system implemented by a set of statements $\mathbb{S}$, we define a function 
$\varphi: F\to 2^{\mathbb{S}}$, where $F$ is the set of features in the 
configurable system. $\varphi(f)$ refers to the set of statements 
that implement feature $f$.
\end{Definition}

In Fig.~\ref{example_bug}, $\varphi($\texttt{SLAB}$)$ is the set of
statements from lines 11--12, while
$\varphi($\texttt{PPC\_256K\_PAGES}$)$ contains only line 4.

\begin{Definition}{({\bf Feature Selection}).}
A feature selection of feature $f$ is a state of feature that is 
either enabled or disabled ($f=T/F$ for short).
\end{Definition}
%
In preprocessor-based program families \cite{kastner2012virtual}, a 
feature selection is controlled by one or more \textit{configuration 
options} (macro symbols). In the example, the feature \texttt{LOCKDEP} 
is controlled by the configuration option \texttt{CONFIG\_LOCKDEP}.
%
In a configurable system, the selections of features are used to specify
particular configurations (variants) of the system.
%
%Entity
%In a program, we are interested in the program entities and the
%operations that are performed on them.
\begin{Definition}{({\bf Program Entity}).}
A program entity is a program element uniquely identified with
a \textit{name} and a \textit{scope}.
\end{Definition}

In our formulation, we are interested in two types of program
entities: {\em variable} and {\em function}.
%
An entity is represented in the form of \texttt{[scope.ent\_name]},
where \texttt{scope} and \texttt{ent\_name} are the scope and the 
name of the program entity respectively. The \textit{scope} and the
\textit{name} of an entity are used together as the identity of the
entity. For example, the code in Fig.~\ref{example_bug} contains
variables \texttt{GLOBAL.kmalloc\_caches}, \texttt{init\_node\_lock\_keys.i}, 
and function \texttt{GLOBAL.init\_lock\_keys}.
%

%Program Operation
In our work, there are two {\bf program operations} on variables and
functions: \textit{define (def)} and \textit{use}. A \textit{def}
operation is used to \textit{assign} a value to a variable or
\textit{define} the body for a function. Meanwhile, a variable is
\textit{used} when it is \textit{used} or \textit{referred} to. A
function is \textit{used} if it is \textit{called}/\textit{invoked} or
used as a function pointer (C/C++). For example, the function
\texttt{GLOBAL.init\_node\_lock\_keys} is \textit{defined} as a set of
statements from lines 21--26 and \textit{called/used} in the statement
at line 33. For a variable \texttt{X}, it might be \textit{declared}
or \textit{destructed}. However, without loss of generality, when
\texttt{X} is declared or destructed, we consider that it is
\textit{defined} with \texttt{UNINIT} or \texttt{UNDEFINED}
respectively. For instance, the value of
\texttt{init\_node\_lock\_keys.i} is \texttt{UNINIT} at line 21.
%
%Implementation Functions

Given a program, $E$ is the set of program entities, and $\mathbb{S}$
is the set of statements of the program, we use 2
\textit{implementation functions} to map each entity in $E$ to the
sets of statements in $\mathbb{S}$ that implement its definitions and
uses:
\begin{Definition}{({\bf Implementation Functions}).}
\begin{itemize}
  \item $Define:E \to 2^\mathbb{S}$, $Define(e)$ is the set of statements that 
  implement the {\bf definitions} of $e$
  \item $Use:E \to 2^\mathbb{S}$, $Use(e)$ is the set of statements that 
  implement the {\bf uses} of $e$
\end{itemize}
\end{Definition}
In Fig.~\ref{example_bug},
$Define($\texttt{GLOBAL.kmalloc\_caches}$)$ contains only statements
at line 16, while, the statement at line 24 is the only element of
$Use($\texttt{GLOBAL.kmalloc\_caches}$)$. For the compound
statements, we also consider the conditional expressions
(\textit{if-statement}, \textit{for-loop}, \textit{while-loop} and
\textit{do-loop}), the control expressions or the options
(\textit{switch-statement}), and take them into account for localizing
bugs.
%
% Break flow

%

%
%In practice, a feature (parent-feature) might contain several other features 
%(child-features). For example,...

%Feature
\begin{Definition}{({\bf Feature-related Functions}).}
In a configurable system, which is implemented by a set of statements 
$S$ and consists of a set of features $F$ and a set of program 
entities $E$, we define two feature-related functions:
\begin{itemize}
	\item $def:F \to 2^E$, $\forall e \in E, e \in def(f) 
	  \Leftrightarrow 	Define(e) $ $\cap \quad \varphi(f) \neq \emptyset$. That is, if the set of statements implementing the definitions of $e$ overlaps with the set of statements realizing the feature $f$, then $e$ belongs to $def(f)$.
          For a program entity, $def(f)$ is the set of program entities which are defined within 
	$f$, e.g., \texttt{cpuup\_prepare.node} $\in def($\texttt{NUMA}$)$.
	%	
	For a function, $f$ is used to realize their definition. For instance, since the statement at line 31 in \texttt{NUMA} is used to define the body of \texttt{GLOBAL.cpuup\_prepare}, this entity is also an element of $def($\texttt{NUMA}$)$.

	\item $use:F \to 2^E$, $\forall e \in E, e \in use(f)
          \Leftrightarrow Use(e) $ $\cap \quad \varphi(f) \neq
          \emptyset$.  That is, if the set of statements~implementing
          the uses of $e$ overlaps with the set of statements realizing the feature $f$, then $e$ belongs to $use(f)$.
          %
	Specifically, $use(f)$ is the set of program entities~that are used within $f$, 
	e.g., \texttt{GLOBAL.PAGE\_SHIFT}, \texttt{GLOBAL.kmalloc\_cach\-es},~\texttt{GLOBAL.slab\_set\_lock\_classes} $\in use($\texttt{LOCKDEP}$)$.
	
%  \item $children: F \to 2^F$, $children(f)$ is the set of features that are 
%  child-features of $f$
%  \item $parent: F \to F$, $parent(f)$ the feature that $f$ is one of child-features of
\end{itemize}
\end{Definition}
%
%
%\textit{The core feature}, $\Omega$, has no parent-feature, while every feature is
%contained by another one. If a (parent) feature is disabled, its child-features 
%are also disabled, and if one of child-features of a feature is enabled, the 
%feature is also enabled.

In a configurable system, two features can interfere each other 
through the program entities that are shared to implement these 
features. In the running example, \texttt{LOCKDEP} \textit{uses} 
\texttt{PAGE\_SHIFT} (line 22) and \texttt{kmalloc\_caches} (line 24) 
that are \textit{defined} by \texttt{SLAB} (line 11) and 
\texttt{!SLOB} (line 16) respectively to implement its functionality.
Hence, the optional functionality offered by \texttt{LOCKDEP} is
influenced when \texttt{SLAB} or \texttt{!SLOB} are enabled/disabled.
%
%
In this work, we focus on the feature interactions through shared 
program entities. The interactions associated with external data such 
as files or database are beyond the scope of this work. If both 
features \textit{use} a program entity, they will not change the 
program's state. By that our assumption, there is no interaction 
between two or more features if they only \textit{use} the same 
program entities. 
%The feature interaction between two features through 
%shared entities and $def$ and $use$ is described by the following 
%definition.

%
%Feature Interaction
\begin{Definition}{({\bf Feature Interaction}).}
The interaction between two features $f_1$ and $f_2$ exits if only if 
$def(f_1) \cap def(f_2) \neq \emptyset$ (called def-def interaction), or $def(f_1) \cap use(f_2) \neq \emptyset$ or $def(f_2) \cap use(f_1) \neq \emptyset$ (called def-use interaction).
\end{Definition}
%

By the above definition, we only represent the interaction between a 
pair of features because the interaction between more than two 
features is still represented by the interactions of pairs features. 
Indeed, assuming that there exists an interaction among $m > 2$ 
features. For simplicity, we consider the case of m = 3, $f_1$, 
$f_2$, and $f_3$, and $E_1$, $E_2$, and $E_3$ are the sets of share 
program entities of three pairs of features ($f_1$, $f_2$), ($f_2$, $f_3$), and ($f_3$, $f_1$).
%
If there exists a common entity shared between all these features, 
the interaction among them is still represented through the interactions 
of the pairs.
%
On the other case, there is no shared entity between all features, 
assuming that $E_1 \cap E_3 = \emptyset$. Since the functionality 
offered by $f_3$ is influenced by $f_1$\footnote{because the roles of 
$f_1$ and $f_3$ in this case are equal} through $f_2$, there exists 
an entity $e_1 \in def(f_1)$ whose value affects another entity $e_3 
\in use(f_3)$ by a value propagation function $p$, such that $e_3 = 
p(e_1)$. Hence, by identifying feature interactions between pairs, 
the interactions among more than two features are still captured. 
%

\begin{Definition}{({\bf Feature Interaction Implementation}).} The implementation of the interaction between features $f_1$ and $f_2$ are the statements that implement the definitions of the shared entities in $f_1$ and $f_2$ (for def-def interactions) and the uses of the shared entities in $f_1$ and $f_2$ (for def-use interactions).
\end{Definition}
\subsection{Feature-interaction detection}
Not decide yet!

\begin{algorithm}
\caption{Feature interaction detection}	
 \label{alg_2}
\begin{algorithmic}[1]
\Procedure {DetectInteractionImplementation}{$F1$, $F2$}
\State $interactionImpl = EmptySet$
\State $DD = intersection(def(F1), def(F2))$
\State $DU = intersection(def(F1), use(F2))$
\State $UD = intersection(use(F1), def(F2))$
\ForAll {$e \in DD$}
	\State $def1 = intersection(def(F1), Define(e))$
	\State $interactionImpl.add(def1)$
	\State $def2 = intersection(def(F2), Define(e))$
	\State $interactionImpl.add(def2)$
\EndFor
\ForAll {$e \in DU$}
	\State $def1 = intersection(def(F1), Define(e))$
	\State $interactionImpl.add(def1)$
	\State $use2 = intersection(use(F2), Use(e))$
	\State $interactionImpl.add(use2)$
\EndFor
\ForAll {$e \in UD$}
	\State $use1 = intersection(use(F1), Use(e))$
	\State $interactionImpl.add(use1)$
	\State $def2 = intersection(def(F2), Define(e))$
	\State $interactionImpl.add(def2)$
\EndFor
\EndProcedure
\Statex
\end{algorithmic}
\end{algorithm}



\subsection{The representation and detection for feature-interaction relevant to pointers and external data}
\textbf{Draft}

The idea is "what are the interested elements (like variable or function) of feature?, and what are operations on the interested elements?"

The interested elements in this case might be: memory location (for aliasing analysis) and external data unit including file and database units. For database units, they might be database, table, row or column.

But anyway, firstly analyzing feature to identify several operations on a set of program entities still need to be performed.