\section{Education Plan - Curriculum Development Activities}
\label{edu-section}

%From an educational point of view, the central goals of the PI's
%educational plan are: 1) to increase the enrollment and success of
%traditionally under-represented students in engineering, especially in
%Security Engineering and Software Engineering, 2) to integrate
%advanced Secure Software Engineering (SSE) practices into
%undergraduate/graduate courses, 3) and to develop a strong
%undergraduate/graduate SSE program.

%Taking advantage of the newly launched BS degree program in Software
%Engineering at Iowa State University, we would like to directly engage
%and improve the Secure Software Engineering practices of local
%companies. ISU's Computer Engineering Department has an excellent
%foundation on Security research via the Information Assurance
%Center~\cite{iac}. This project will be served as a connection that
%bridges the gap between Information Assurance education and Software
%Engineering education at ISU. The project will strengthen both
%programs.

\noindent {\bf Undergraduate Software Engineering Education.} PI
Nguyen is one of the key SE faculty members in the BS, MS, and Ph.D.
degree programs in Software Engineering (SE) at University of Texas at
Dallas. He has also a track record in contributing to Undergraduate
Education Program when he was at Iowa State University. He has
successfully introduced several courses including Software
Architecture and Design (CprE339) and Software Project Management
(CprE329). Taking advantage of this project, we will introduce a new
course in NLP+SE (natural language processing + software
engineering). The key teaching philosophy in this course is the
combination of theory and practice in which students will be
introduced different principles and theories in the application of NLP
in SE artifacts. The tentative modules include 1) basic principles,
processes, and paradigms in NLP, 2) programming languages versus
natural languages, 3) API usages and reuse, 4) Statisical models used
in SE applications, 5) machine translation and code migration, 6)
language models for source code, 7) applications of machine
translation in SE, 8) word embeddings and SE applications, 9) deep
learning and SE applications, etc.

\noindent {\bf Graduate Software Engineering Education} The PI works
with other UTD faculty to develop a series of six to seven SE graduate
courses that will be offered at least every semester. This year, the
PI introduced a new graduate level course on ``Software Mining and
Analysis''. The PI will introduce a new graduate course on the topic
of NLP+SE. Tools developed by this research will be used in class
projects. The course will focus on several SE advanced methods that
aim to help advance software engineering with NLP techniques. The
tentative topics include 1) bug report processing, 2) source code
analysis with NLP, 3) cross-language analysis between texts and code,
4) code and text retrieval in SE applications, 5) NLP techniques and
type inference, 6) NLP and de-ofuscation, 7) bug-fixing and
machine learning, 8) NL+SE for big data, etc.


%%%Some of these modules have been introduced to students at ISU as part
%%%of the BS Software Engineering degree program and several programs at
%%%the Information Assurance center\cite{iac}.

%The results from the security domain analysis will be used in all five
%components. The proposed RSD method, SSADL specification, and design
%supporting tools will be used in the teaching components in
%(B). 

%Efforts will be made to involve minority graduate students recruited
%by ISU through GEM and NSF AGEP programs. Currently, PI serves as a
%mentor for the Leadership through Engineering Academic Diversity
%(LEAD) programs which aims to enrich the educational experience of
%minority engineering students on the ISU campus. Participants in ISU's
%Women in Science and Engineering will be given the opportunity to
%participate in the proposed research. Annually, there are Security
%Summer Camps for high-school students and faculty, organized by ISU's
%Information Assurance Center. These events allow us to involve
%minority students and faculty into the proposed research.

%TODO
%\input{dissemination.tex}
