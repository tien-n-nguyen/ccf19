%Son, optional segment of code --> feature
In practice, a configurable system usually provides a large number of
{\bf configuration options} to configure several optional segments of code
to be present or absent, in addition to
\textit{the core} of the system.
%
Those optional segments of code are aimed to realize the
optional \textbf{features} of the system.
%
For example, in Linux Kernel, the configuration options have the
prefix of \texttt{CONFIG\_}, and they can have different
values. Without loss of generality, we assume that the value of a
configuration option is either \texttt{true(T)} or
\texttt{false(F)} (We can consider the entire conditional
  expressions of non-boolean options as boolean ones, e.g.,
  \texttt{CONFIG\_A>10} as \texttt{CONFIG\_A>10=T/F}).

\begin{Definition}{({\bf Configuration Option}).}
A configuration option ({\em option} for short) is an element that is used
to configure the source code of a configurable system, such that the
option's value determines the presence or absence of one or more
segments of~code.
\end{Definition}

In a configurable system, the presence or absence of code segments is
dependent on the values of multiple options. In Figure~\ref{example1},
the lines 19 and 20 are presented only when both
\texttt{CONFIG\_TWL4030\_CORE} and \texttt{CONFIG\_OF\_IRQ} are
\texttt{T}. Thus, at line 19, \texttt{irq\_domain\_simple\_ops} is
potentially used to assign as a value to the variable \texttt{ops}
when both above options are \texttt{T}.

%\texttt{CONFIG\_OF\_IRQ} is \texttt{true}.

\begin{Definition}{({\bf Selection Functions}).}
In a configurable system, we define selection functions as the
functions from $O\times V$ to $2^P$, where $O$ is the set of
configuration options, $V=\{\texttt{T, F}\}$, and $P$ is the set of program
entities used in the code of the configurable system. We define four selection functions:

\begin{itemize}[leftmargin=4mm]

\item $\alpha: O \times V \to 2^P$, $\alpha(o, v) = D$, where $o \in O, v \in \{\texttt{T, F}\}$, and $D$ is the set of entities potentially {\bf declared} if $o = v$.

\item $\beta: O \times V \to 2^P$, $\beta(o, v) = D$, where $o \in O, v \in \{\texttt{T, F}\}$, and $D$ is the set of entities potentially {\bf assigned} if $o = v$.

\item $\gamma: O \times V \to 2^P$, $\gamma(o, v) = D$, where $o \in O, v \in \{\texttt{T, F}\}$, and $D$ is the set of entities potentially {\bf used} if $o = v$.

\item $\delta: O \times V \to 2^P$, $\delta(o, v) = D$, where $o \in O, v \in \{\texttt{T, F}\}$, and $D$ is the set of entities potentially {\bf destructed} if $o = v$.
\end{itemize}

\end{Definition}

For example, in Figure~\ref{example1}:

\begin{itemize}

\item $\alpha($\texttt{CONFIG\_SPARC, F}$)$=$\{$\texttt{GLOBAL.of\_platform\_populate, of\_platform\_populate.node}$\}$

\item $\beta($\texttt{CONFIG\_OF\_IRQ, T}$)$=$\{$\texttt{twl\_probe.ops}$\}$

\item $\gamma($\texttt{CONFIG\_OF\_IRQ, T} $)$=$\{$\texttt{GLOBAL.irq\_domain\_simple\_ops}$\}$

\end{itemize}

\begin{Definition}{({\bf Configuration}).}
Given a configurable system, a configuration is a specific
selection of configuration options, which defines a \textbf{variant}
of the system.
\end{Definition}

Configuration options are used to control the features that are
represented by certain segments of code. For example, in
Figure~\ref{example1}, the feature represented by the segment of code
\texttt{X} (feature \texttt{X}) is enabled if the value of
the configuration option \texttt{CONFIG\_IRQ\_DOMAIN} is \texttt{true},
whereas feature \texttt{Y} is enabled if both \texttt{CONFIG\_OF\_IRQ} and \texttt{CONFIG\_\-TWL4030\_CORE} are \texttt{true}.

\begin{Definition}{({\bf Feature}).}
In a configurable system, a feature $f$ is implemented by applying
program operations on a set of program entities, whose
presence/absence is controlled by certain configuration options.
%
%In the code, a feature is controlled by certain configuration options.
%
We denote it by $f \sim OP \times \rho$ where $OP$ is the set of
program operations and $\rho$ is the set of program entities.
\end{Definition}

A special case of features is that $f$ is \textit{the core feature}
($F$), $A \cup B \cup \Gamma \cup \Delta=\rho$, where $A, B, \Gamma,
\Delta$ are the sets of program entities that are declared, assigned,
used and destructed in \textit{the core system}. $F$ is not controlled
by any configuration option.

%In fact, two different features might have similar these attributes.

% argument on feature interactions via program entities

%\vspace{0.03in}

%\subsection{Feature Interaction}

\noindent {\bf Feature Interaction.}
In a configurable system, a feature can influence or modify (often
called \textit{interact} with) the functions offered by other
features through shared program entities that are used to implement
the features. For example, features \texttt{X}, \texttt{K} and \texttt{Z} 
interact with one another via the variables \texttt{GLOBAL.irq\_domain\_} \texttt{simple\_ops} and \texttt{twl\_probe.temp}.
%
The manners the features interacting with each other depend how the
shared program entities are operated. For example, feature \texttt{Y}
\textit{assigns} \texttt{\&irq\_domain\_simple\_ops} to \texttt{ops}
and feature \texttt{K} \textit{uses} that variable (line 22). If no
assignment was done in \texttt{Y}, the dereferencing in \texttt{K}
would be invalid, causing a bug.

%tien

We present only on the interactions between pairs of features because
the interactions between more than two features can be modeled as the
operations on the shared variables between pairs of features.  To
proof this statement we assume that there exists an interaction among
$m > 2$ features. For simplicity, we consider the case of $m=3$, and
the interaction among $f_1\sim OP \times \rho_1, f_2\sim OP \times
\rho_2$ and $f_3\sim OP \times \rho_3$. There are two cases of this
interaction. First, there exists an entity that shared by all 3
features, $\rho_1 \cap \rho_2 \cap \rho_3 = \omega \neq \emptyset$. In
this case, since $\rho_1 \cap \rho_2 \supset \omega$ and $\rho_2
\cap \rho_3 \supset \omega$, identifying interactions between pairs
directly captures the interaction among 3 features. The second case is
that $\rho_1 \cap \rho_2 = \omega_1$, $\rho_2 \cap \rho_3 = \omega_2$
and $\omega_1 \cap \omega_2 = \emptyset$. Meanwhile, $f_3$ is
influenced by $f_1$ (because the roles of $f_1$ and $f_3$
  features in this case are equal). This leads to that there exist
entities: $e_1 \in \omega_1, e_2 \in \omega_2$, such that $e_2 =
p(e_1)$, where $p$ is a value propagation function. This means the
value of $e_1$ is propagated to $e_2$, and that influences
$f_3$. Hence, the interaction among 3 features is still captured by
determining interactions between pairs of features.

For instance, the interaction among features \texttt{X}, \texttt{K} and \texttt{Z} can be broken
down into the shared program entities between two pairs of features as
follows: (\texttt{X}, \texttt{K}) via the variable \texttt{GLOBAL.irq\_domain\_\-simple\_ops}, and
(\texttt{K}, \texttt{Z}) via the variable \texttt{twl\_probe.temp}.
%
%Detecting feature interactions related 3rd-party features is beyond the scope of this work.
%
%
Thus, our solution can still model the interactions with more than two
features via the operations on their shared program entities.
%
From now on, we refer to a feature interaction as an
interaction determined via the shared program entities between
a pair of features.

%The manners that pairs of features interact with each other depend how
%they operate on the shared program entities. For example, feature $Y$
%\textit{assgigns} \texttt{\$irq\_domain\_simple\_ops} to \texttt{ops}
%and feature $K$ \textit{uses} that variable (line 25).

We focus on the feature interaction through the shared
program entities. The feature interactions when the variables are
associated with the external data such as when they interfere with
each other's behaviors on files or databases are beyond the scope of
our static analysis-based solution. Similarly, we will not detect the
interactions through pointers or arrays in this work.  As a
consequence, if both features {\em use} (refer to) a program entity,
they will not change the program's state. Thus, there is no
interaction between two features if they only use shared functions and
variables.

%Note that a program entity is not considered as aconnection between features if these features only \textit{use} the entity. For example, \texttt{of\_platform\_populate} is not considered as a connecting entity between $K$ and $Z$ because this function is only used in these features. In a program, \textit{using} variables that changes the state of the program only when variables associate with external data such files, databases or are arrays or pointers. Evaluating the suspiciousness related to bugs on external data, array or pointer is beyond the scope of this work. Thus, there is no interaction between features if they only use shared functions and variables.
%

The interactions between two features $f_1 \sim OP \times \rho_1$, and
$f_2 \sim OP \times \rho_2$ with $\rho_1 \cap \rho_2 \neq \emptyset$,
can be categorized into nine following kinds of interactions (the
\textit{use-use} case is eliminated):

\begin{itemize}[leftmargin=4mm]

\item $\exists e \in \rho_1 \cap \rho_2$, $e$ is declared in both $f_1$ and $f_2$ (\textit{declare-declare})

\item $\exists e \in \rho_1 \cap \rho_2$, $e$ is declared in $f_1$ and then assigned in $f_2$ (\textit{declare-assign})

\item $\exists e \in \rho_1 \cap \rho_2$, $e$ is declared in $f_1$ and used in $f_2$ (\textit{declare-use})

\item $\exists e \in \rho_1 \cap \rho_2$, $e$ is declared in $f_1$, and destructed in $f_2$ (\textit{declare-destruct})

\item $\exists e \in \rho_1 \cap \rho_2$, $e$ is assigned in both $f_1$ and $f_2$ (\textit{assign-assign})

\item $\exists e \in \rho_1 \cap \rho_2$, $e$ is assigned in $f_1$ and used in $f_2$ (\textit{assign-use})

\item $\exists e \in \rho_1 \cap \rho_2$, $e$ is assign in $f_1$ and destructed in $f_2$ (\textit{assign-destruct})

\item $\exists e \in \rho_1 \cap \rho_2$, $e$ is used in $f_1$ and destructed in $f_2$ (\textit{use-destruct})

\item $\exists e \in \rho_1 \cap \rho_2$, the entity is destructed in both $f_1$ and $f_2$ (\textit{destruct-destruct})


\end{itemize}


\noindent {\bf Feature Interaction Detection.} In a configurable
system, features (except \textit{the features} in the core of the
system) are controlled by certain configuration options. Thus, if
there exists an interaction among features, the interaction will
be:

\begin{itemize}[leftmargin=4mm]

\item \textit{declare-declare}, there exist two options $o_1, o_2$ and their values $v_1, v_2$, such that $\alpha(o_1, v_1) \cap \alpha(o_2, v_2) \neq \emptyset$

\item \textit{declare-assign}, there exist two options $o_1, o_2$ and their values $v_1, v_2$, such that $\alpha(o_1, v_1) \cap \beta(o_2, v_2) \neq \emptyset$

\item \textit{declare-use}, there exist two options $o_1, o_2$ and their values $v_1, v_2$, such that $\alpha(o_1, v_1) \cap \gamma(o_2, v_2) \neq \emptyset$,

\item \textit{declare-destruct}, there exist two options $o_1, o_2$ and their values $v_1, v_2$, such that $\alpha(o_1, v_1) \cap \delta(o_2, v_2) \neq \emptyset$

\item \textit{assign-assign}, there exist two options $o_1, o_2$ and their values $v_1, v_2$, such that $\beta(o_1, v_1) \cap \beta(o_2, v_2) \neq \emptyset$

\item \textit{assign-use}, there exist two options $o_1, o_2$ and their values $v_1, v_2$, such that $\beta(o_1, v_1) \cap \gamma(o_2, v_2) \neq \emptyset$

\item \textit{assign-destruct}, there exist two options $o_1, o_2$ and their values $v_1, v_2$, such that $\beta(o_1, v_1) \cap \delta(o_2, v_2) \neq \emptyset$

\item \textit{use-destruct}, there exist two options $o_1, o_2$ and their values $v_1, v_2$, such that $\gamma(o_1, v_1) \cap \delta(o_2, v_2) \neq \emptyset$

\item \textit{destruct-destruct}, there exist two options $o_1, o_2$ and their values $v_1, v_2$, such that $\delta(o_1, v_1) \cap \delta(o_2, v_2) \neq \emptyset$

\end{itemize}