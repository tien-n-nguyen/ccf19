\textbf{SON: INTRO THE PROBLEM}

In practice, a configurable system usually provides a large number of
{\bf configuration options} to configure several optional segments of code
to be present or absent, in addition to
\textit{the core} of the system.
%
Those optional segments of code are aimed to realize the
optional \textbf{features} of the system.
%
For example, in Linux Kernel, the configuration options have the
prefix of \texttt{CONFIG\_}, and they can have different
values. Without loss of generality, we assume that the value of a
configuration option is either \texttt{true(T)} or
\texttt{false(F)} (We can consider the entire conditional
  expressions of non-boolean options as boolean ones, e.g.,
  \texttt{CONFIG\_A>10} as \texttt{CONFIG\_A>10=T/F}).

%Code in example_src.c
\begin{figure}
\centering
\includegraphics[width=0.5\textwidth]{example_bk.png}
\caption{A simplified buggy version of Linux kernel}
\label{example1}
\end{figure}


\begin{Definition}{({\bf Configuration Option}).}
A configuration option ({\em option} for short) is an element that is used
to configure the source code of a configurable system, such that the
option's value determines the presence or absence of one or more
segments of~code.
\end{Definition}

In a configurable system, the presence or absence of code segments is
dependent on the values of multiple options. In Figure~\ref{example1},
the lines 19 and 20 are presented only when both
\texttt{CONFIG\_TWL4030\_CORE} and \texttt{CONFIG\_OF\_IRQ} are
\texttt{T}. Thus, at line 19, \texttt{irq\_domain\_simple\_ops} is
potentially used to assign as a value to the variable \texttt{ops}
when both above options are \texttt{T}.

%\texttt{CONFIG\_OF\_IRQ} is \texttt{true}.

\begin{Definition}{({\bf Selection Functions}).}
In a configurable system, we define selection functions as the
functions from $O\times V$ to $2^P$, where $O$ is the set of
configuration options, $V=\{\texttt{T, F}\}$, and $P$ is the set of program
entities used in the code of the configurable system. We define four selection functions:

\begin{itemize}

\item $\alpha: O \times V \to 2^P$, $\alpha(o, v) = D$, where $o \in O, v \in \{\texttt{T, F}\}$, and $D$ is the set of entities potentially {\bf declared} if $o = v$.

\item $\beta: O \times V \to 2^P$, $\beta(o, v) = D$, where $o \in O, v \in \{\texttt{T, F}\}$, and $D$ is the set of entities potentially {\bf assigned} if $o = v$.

\item $\gamma: O \times V \to 2^P$, $\gamma(o, v) = D$, where $o \in O, v \in \{\texttt{T, F}\}$, and $D$ is the set of entities potentially {\bf used} if $o = v$.

\item $\delta: O \times V \to 2^P$, $\delta(o, v) = D$, where $o \in O, v \in \{\texttt{T, F}\}$, and $D$ is the set of entities potentially {\bf destructed} if $o = v$.
\end{itemize}

\end{Definition}

For example, in Figure~\ref{example1}:

\begin{itemize}

\item $\alpha($\texttt{CONFIG\_SPARC, F}$)$=$\{$\texttt{GLOBAL.of\_platform\_populate, of\_platform\_populate.node}$\}$

\item $\beta($\texttt{CONFIG\_OF\_IRQ, T}$)$=$\{$\texttt{twl\_probe.ops}$\}$

\item $\gamma($\texttt{CONFIG\_OF\_IRQ, T} $)$=$\{$\texttt{GLOBAL.irq\_domain\_simple\_ops}$\}$

\end{itemize}

\begin{Definition}{({\bf Configuration}).}
Given a configurable system, a configuration is a specific
selection of configuration options, which defines a \textbf{variant}
of the system.
\end{Definition}

Configuration options are used to control the features that are
represented by certain segments of code. For example, in
Figure~\ref{example1}, the feature represented by the segment of code
\texttt{X} (feature \texttt{X}) is enabled if the value of
the configuration option \texttt{CONFIG\_IRQ\_DOMAIN} is \texttt{true},
whereas feature \texttt{Y} is enabled if both \texttt{CONFIG\_OF\_IRQ} and \texttt{CONFIG\_\-TWL4030\_CORE} are \texttt{true}.


The interactions between two features $f_1 \sim OP \times \rho_1$, and
$f_2 \sim OP \times \rho_2$ with $\rho_1 \cap \rho_2 \neq \emptyset$,
can be categorized into nine following kinds of interactions (the
\textit{use-use} case is eliminated):

\begin{itemize}

\item $\exists e \in \rho_1 \cap \rho_2$, $e$ is declared in both $f_1$ and $f_2$ (\textit{declare-declare})

\item $\exists e \in \rho_1 \cap \rho_2$, $e$ is declared in $f_1$ and then assigned in $f_2$ (\textit{declare-assign})

\item $\exists e \in \rho_1 \cap \rho_2$, $e$ is declared in $f_1$ and used in $f_2$ (\textit{declare-use})

\item $\exists e \in \rho_1 \cap \rho_2$, $e$ is declared in $f_1$, and destructed in $f_2$ (\textit{declare-destruct})

\item $\exists e \in \rho_1 \cap \rho_2$, $e$ is assigned in both $f_1$ and $f_2$ (\textit{assign-assign})

\item $\exists e \in \rho_1 \cap \rho_2$, $e$ is assigned in $f_1$ and used in $f_2$ (\textit{assign-use})

\item $\exists e \in \rho_1 \cap \rho_2$, $e$ is assign in $f_1$ and destructed in $f_2$ (\textit{assign-destruct})

\item $\exists e \in \rho_1 \cap \rho_2$, $e$ is used in $f_1$ and destructed in $f_2$ (\textit{use-destruct})

\item $\exists e \in \rho_1 \cap \rho_2$, the entity is destructed in both $f_1$ and $f_2$ (\textit{destruct-destruct})


\end{itemize}


\noindent {\bf Feature Interaction Detection.} In a configurable
system, features (except \textit{the features} in the core of the
system) are controlled by certain configuration options. Thus, if
there exists an interaction among features, the interaction will
be:

\begin{itemize}[leftmargin=4mm]

\item \textit{declare-declare}, there exist two options $o_1, o_2$ and their values $v_1, v_2$, such that $\alpha(o_1, v_1) \cap \alpha(o_2, v_2) \neq \emptyset$

\item \textit{declare-assign}, there exist two options $o_1, o_2$ and their values $v_1, v_2$, such that $\alpha(o_1, v_1) \cap \beta(o_2, v_2) \neq \emptyset$

\item \textit{declare-use}, there exist two options $o_1, o_2$ and their values $v_1, v_2$, such that $\alpha(o_1, v_1) \cap \gamma(o_2, v_2) \neq \emptyset$,

\item \textit{declare-destruct}, there exist two options $o_1, o_2$ and their values $v_1, v_2$, such that $\alpha(o_1, v_1) \cap \delta(o_2, v_2) \neq \emptyset$

\item \textit{assign-assign}, there exist two options $o_1, o_2$ and their values $v_1, v_2$, such that $\beta(o_1, v_1) \cap \beta(o_2, v_2) \neq \emptyset$

\item \textit{assign-use}, there exist two options $o_1, o_2$ and their values $v_1, v_2$, such that $\beta(o_1, v_1) \cap \gamma(o_2, v_2) \neq \emptyset$

\item \textit{assign-destruct}, there exist two options $o_1, o_2$ and their values $v_1, v_2$, such that $\beta(o_1, v_1) \cap \delta(o_2, v_2) \neq \emptyset$

\item \textit{use-destruct}, there exist two options $o_1, o_2$ and their values $v_1, v_2$, such that $\gamma(o_1, v_1) \cap \delta(o_2, v_2) \neq \emptyset$

\item \textit{destruct-destruct}, there exist two options $o_1, o_2$ and their values $v_1, v_2$, such that $\delta(o_1, v_1) \cap \delta(o_2, v_2) \neq \emptyset$

\end{itemize}