\subsubsection{Neccesity and Sufficiency Conditions}

We aim to detect the set of features along with their~feature
selections that have potential interactions and failed the~tests.  For
a buggy system, whose sets of passed and failed configurations are
denoted by $CP$ and $CF$ respectively, we~aim to detect the {\em
  optimal set of feature selections}, called {\em suspicious
  partial configuration} ($SPC$) of the fault(s) that must satisfy
both necessity and sufficiency conditions.

%For the neccesity condition, for any strict subset of $SPC$, there
%exists a configuration that contains it and does not fail any
%test.

%For the sufficiency condition, the selections of other features
%outside of the feature selections in $SPC$ do not have any impact on
%the test results.

\begin{Definition}{({\bf Necessity}).}
$\forall SPC', SPC' \subsetneq SPC$, such that $\exists c \in CP$, $c
  \supset SPC'$.
\end{Definition}

That is, for the necessity condition, for any strict subset of $SPC$,
there exists a configuration that contains it and does not fail any
test.

From the definition, we have that {\em $SPC'$ does not cause the
  fault(s)} because otherwise, with $c$ containing $SPC'$, $c$ and
$SPC'$ would fail at least one test, which contradicts to the fact
that $c$ passes all the tests.

{\bf We need the necessity condition} {\em because if $SPC$ does not
  satisfy that condition, then \textbf{SSSS} would have to consider
  $SPC'$ that does not cause the fault(s).} That is, \textbf{SSSS}
would unnecessarily consider $SPC'$.
%, that does not cause the bug(s).

%$c$ contains $SPC'$ and $c$ passed all the tests, therefore, {\em
%  $SPC'$ does not cause the fault(s)} (If it did, $c$ and itself
%would fail at least one test).

%{\bf We need the necessity condition} {\em because if $SPC$
%  does not satisfy that condition, then our algorithm will have to
%  consider the strict subset $SPC'$ of $SPC$ that does not cause the
% fault(s).}

From the necessity condition, we have the following:

\begin{Corollary}
$SPC$ must satisfy: $\forall c \in \mathbb{C}$, $c \supset SPC \implies c \in CF$, where $\mathbb{C}$ is the set of all configurations. That is, any configuration $c$ containing $SPC$ fails at least a test.
\end{Corollary}

%This is because if $c$ does not fail any test, $SPC$ cannot satisfy
%the necessity condition.

\begin{Definition}{({\bf Sufficiency}).}
$\forall SPC', SPC' \supsetneq SPC$, such that $\forall c \in \mathbb{C}$, $c \supset SPC' \implies c \in CF$.
\end{Definition}

%Every configuration that contains the fault(s) fails at least one
%test.

For the sufficiency condition, the selections of other features
outside of the feature selections in $SPC$ do not change the test
results. That is, for any strict superset $SPC'$ of $SPC$, any
configuration containing $SPC'$ would fail at least one test.

{\bf We need sufficiency condition} otherwise, we will
unnecessarily consider the other feature selections outside of $SPC$.
%

\noindent {\bf Proof.} Let us prove this by contradiction. Assuming
that there exits another {\em necessary} $SPC'$ such that $SPC'
\supsetneq SPC$, and we consider the feature selections in $SPC'$.
%
Due to the necessity condition, we have $\forall PC \subsetneq SPC'$,
$\exists c \in CP$, $c \supset PC$.
%
Let us apply this formula where $PC$ = $SPC$, which is a subset of $SPC'$.
We have $\exists c \in CP$, $c \supset SPC$, and contradicts
with Corollary~1 ($c$ must fail at least one test).
%
Therefore, $SPC$ must satisfy the sufficiency condition.

%=============================================
%\begin{itemize}[leftmargin=4mm]
%\item \textbf{Necessity}: $\forall SPC', SPC' \subsetneq SPC$, such that $\exists c \in CP$, $c \supset SPC'$.

%Because if there exits a passed configuration that contains all $SPC'$, $SPC'$ \textbf{dose not} cause the bug.

%\textbf{Corollary}: $SPC$ must satisfy: $\forall c \in \mathbb{C}$, $c \supset SPC \implies c \in CF$, where $\mathbb{C}$ is the configuration space of the system (1).

%\item \textbf{Sufficiency}: $\forall SPC', SPC' \supsetneq SPC$, such that $\forall c \in \mathbb{C}$, $c \supset SPC' \implies c \in CF$.

%Indeed, assuming that there exits another \textbf{necessary} $SPC'$, $SPC' \supsetneq SPC$. We have $\forall PC \subsetneq SPC'$, $\exists c \in CP$, $c \supset PC$, which leads to $\exists c \in CP$, $c \supset SPC$, and contradicts with (1). Therefore, $SPC$ must satisfy the sufficiency condition.

%\end{itemize}
%==============================================

%OLD
%This section describes our algorithm to detect the minimum sets of features whose enabling and disabling can change the test results, given the test results for each configuration (e.g., Table~\ref{test_resutls}).
%
%Intuitively, for a particular failed configuration, such features are the features that when we enable/disable each of them, and then the resulting configuration is passed.
%
%We call a set of the selections of such features in the failed configuration is a \textit{suspicious partial configuration} (SPC). SPC is a candidate for the \textit{fault-causing partial configuration} of the \textit{configuration-dependent fault} in that configuration.

%\input{sufficient_necessary}

%\subsection{Formulation}

% $\mathbb{C}$ is the configuration space

%CP and CF are the set of passed configurations and the set of failed configurations repspectively.

%PC is a set of feature selections


%Sufficient: $\forall PC', PC' \supsetneq PC$, such that $\forall c \in \mathbb{C}$, $c \supset PC'  \implies c \in CF$


%Necessary: $\forall PC', PC' \subsetneq PC$, such that $\forall c \in \mathbb{C}$, $c \supset PC' \implies c \in CP$


%Configuration-dependent Fault
%\begin{definition}{({\bf Configuration-dependent Fault}).}
%In a configurable system, a fault of the system is considered as a
%configuration-dependent bug if only if there are two sets of
%configurations $CP$ and $CF$, where $CP, CF \subsetneq \mathbb{C}$, such
%that all configurations in $CP$ execute as expected, while all
%configurations in $CF$ reveal the bug.
%\end{definition}
%
%For example, the bug in the Fig.~\ref{example_bug} is revealed only in $c_2$ and $c_7$ in which the features \texttt{LOCKDEP}, \texttt{PPC\_256K\_PAGES}, \texttt{\!SLOB}, and \texttt{SLAB} are all
%enabled and \texttt{PPC\_16K\_PAGES} is disabled.
%
%In highly configurable systems, most of reported bugs  are
%configuration-dependent faults that are only exposed under a set of
%certain configurations \cite{Abal:2018} \cite{Garvin:2011}.

\subsubsection{Important Property to Build the $SPC$ set}

%OLD ==================================================
%In fact, in a configurable code containing a configuration-dependent
%bug, not all features are relevant to the visibility of the bug. Yet,
%there are certain features that are enabled or disabled, in which {\em
%  the bug is revealed during executing/testing every configuration
%  containing these feature selections, regardless of the selections of
%  other features}. For example, in Fig.~\ref{example_bug}, the bug is
%only triggered by the set of enabled features including
%\texttt{LOCKDEP}, \texttt{PPC\_256K\_PAGES}, \texttt{\!SLOB}, and
%\texttt{SLAB}, along with the disabling of \texttt{PPC\_16K\_PAGES},
%regardless of \texttt{NUMA}. The bug is invisible if
%\texttt{PPC\_16K\_PAGES} is enabled.

%\begin{definition} \label{fault-causing-partial-config-def} {({\bf Fault-causing Partial Configuration}).}
%Given a configurable system containing configuration-dependent bugs,
%the fault-causing partial configuration $PC$ of a particular
%configuration-dependent bug is the non-empty minimal set of feature
%selections, such that $\forall c \in \mathbb{C}, c \supseteq PC
%\implies c \in CF$, where $CF$ is the set of configurations that
%reveal the bug.
%\end{definition}

%For example, the fault-causing partial configuration of the bug in
%Fig.~\ref{example_bug} is \{\texttt{SLAB=T},
%\texttt{PPC\_16K\_PAGES=F}, \texttt{PPC\_256K\_PAGES=T},
%\texttt{LOCKDEP=T}, \texttt{\!SLOB=T}\}.
%=============================================================

Before presenting how we construct $SPC$, let us explain the
following important property of $SPC$ that is useful in such
construction. From the above conditions, we have

%By Definition~\ref{fault-causing-partial-config-def}, we have
%following property that is useful to build the suspicious partial
%configuration.

%\begin{property}
%\label{proper_1}
%For a failed configuration $c \in CF$ that contains a fault-causing
%partial configuration $PC$, $\forall s_1, s_2 \neq \emptyset$, $s_1 = S
%\backslash s_2$, if $\exists c' \in CP$, such that $s_1 \subset c'$,
%then $s_2 \cap PC \neq \emptyset$.
%\end{property}
%
%Indeed, if the subset $s_1$ of $c$, is a part of a passed configuration, $c' \in CP, s_1 \subset c'$, then is not fully contained by $s_1$, $PC \not \subseteq s_1$, because if $PC \subseteq s_1$, $c' \supset PC$ and $c'$ is failed. This leads to a part or the whole of $PC$ must be contained by $s_2$. If $s_2$ contains only one selection $s$, then $s \in P$.

%\begin{property}
%\label{proper_2}
%For a failed configuration $c \in CF$ that contains a fault-causing
%partial configuration $PC$, if $\forall s \in c$, $\exists c' \in CP$,
%such that $c \backslash s \subset c'$, then $s \in PC$.
%\end{property}

\begin{Property}
\label{proper_2}
For a failed configuration $c \in CF$ that contains the suspicious
partial configuration $SPC$, if a feature $f$ of selection $s \in c$
is changed to $s' \neq s$, and the resulting configuration $c' =
\{s'\}+c \backslash \{s\}$ is passed, then $s \in SPC$. That is, if we
switch a selection of a feature $f$ in a failed configuration $c$ and
the resulting configuration $c'$ now passes all the tests, then that
selection of $f$ must be in the $SPC$ set.
\end{Property}

\noindent {\bf Proof.} Indeed, if $S_1 = c \backslash \{s\}$ is a part
of a passed configuration, $c'$, and $S_1 \subset c'$, then $SPC \not
\subseteq S_1$. Meanwhile, $SPC \subseteq c$, which leads to $c
\backslash {S_1 }\cap SPC \neq \emptyset$, $c \backslash {S_1}=\{s\}$,
and $s \in SPC$.
%

In the running example, for the failed configuration $c_2$, the
selection $s=$\texttt{(SLAB=T)}, $c_2 \backslash s \subset c_4$, $c_4
\in SPC$, and \texttt{(SLAB=T)} is an element of the $SPC$ contained
by $c_2$. In general, the principle of the examining if a selection is
an element of $SPC$ in property \ref{proper_2} can be applied for any
set of selections that contains $SPC$, not only failed configurations.
%

\subsubsection{Detailed Algorithm to Build the $SPC$ Set}

In the general cases, we use Property \ref{proper_2} to build the $SPC$
of configuration-dependent bugs. Using this property, we can start
with each feature selection that makes the bugs visible in each failed
configuration, and expand it gradually.
%

In the case that system contains a single bug, all configurations in
$CF$ are failed by the same bug. This leads to the $SPC$ of the single
bug in the system is the common subset of all failed configurations,
$SPC \subseteq \bigcap {c_i}$, where $c_i \in CF$.

\begin{algorithm}
\caption{\textbf{SSSS} Suspicious partial configuration detection algorithm for single bug}	
 \label{alg_1}
\begin{algorithmic}[1]
\Procedure {DetectSPC}{$CP$, $CF$}
\State $SPC = EmptySet$
\State $examinedSels = commonSubSet(CF)$
%\State $unrelatedFeatures = FindUnrelatedFeatures(CF)$ 

%\ForAll {$f \in unrelatedFeatures$}
%	\State $enabledSel = enable(f)$
%	\State $disabledSel = disable(f)$
%	\State $remove(consideredSels, enabledSel)$
%	\State $remove(consideredSels, disabledSel)$
%\EndFor

\ForAll {$s \in examinedSels$}
	\State $subset = examinedSels.subtract(s)$
	\ForAll {$passedConfig \in CP$}
		\If {$passedConfig.containsAll(subset)$}
			\State $SPC.add(s)$
			\State $break$
		\EndIf
	\EndFor
\EndFor
\EndProcedure
\Statex

%\Procedure {FindUnrelatedFeatures}{$CF$}
%\State $unrelatedFeatures = EmptySet$
%
%\ForAll {$f \in AllFeatures$}
%	\State $enabledSel = enable(f)$
%	\State $disabledSel = disable(f)$
%	\State $config = findContainingConfig(enabledSel, CF)$
%	\State $other = findContainingConfig(disabledSel, CF)$
%	\If {$config \neq NULL \wedge other \neq NULL$ }
%		\State $addToSet(f, unrelatedFeatures)$
%	\EndIf
%\EndFor
%\EndProcedure
%\Statex
\end{algorithmic}
\end{algorithm}


In this version of \textbf{SSSS}, we implement the algorithm to detect
suspicious partial configuration of single bug in configurable system
(Algorithm \ref{alg_1}), given the test results ($CP$ and $CF$), based
on Property~\ref{proper_2}. In Algorithm~\ref{alg_1}, to detect the
suspicious partial configuration, $SPC$, the set of all
selections that need to be examined ($examinedSels$) is identified by
finding the common subset of all failed configurations in $CF$ (line
3). Next, to construct $SPC$, each selection in
$examinedSels$ is examined whether it is an element of the
suspicious partial configuration (lines 4--9).
%
For the running example, the detected suspicious partial configuration
by Algorithm~\ref{alg_1} is \{\texttt{PPC\_16K\_PAGES=F},
\texttt{SLAB=T}, \texttt{PPC\_256K\_PAGES=T}, \texttt{LOCKDEP=T},
\texttt{!SLOB=T}\}.