\subsection{Research Objectives and Anticipated Results}
\label{research-objectives-section}

%which combines the broad coverage of code search methods with the
%convenience of code synthesis methods.  This will allow us to create
%developer assistance tools that can synthesize previously unseen code
%snippets from NL descriptions, over a wide variety of potential
%developer queries, in many languages.

This is a proposal is to investigate and develop {\bf a scientific
  foundation and a novel methodology for API usage code synthesis
from NL queries to accelerate API usability}.
%
The results will enable us to create developer assistance tools that
can synthesize previously unseen code snippets from NL descriptions,
over a wide variety of potential developer queries, in any application
domain. First, our key philosophy in this project is the {\bf
  regularity/repetitiveness of API usages}. Based on our previous NSF
research project ``Exploiting the Naturalness of Software'', Hindle
{\em et al.}~\cite{hindle-icse12} has shown that software exhibits its
naturalness: source code has a higher degree of regularity than
English texts. We expect the same principle of naturalness of software
to occur for the API elements in the API code usages. API elements
including API classes, method calls, and field accesses of software
libraries do not occur randomly. They appear regularly with certain
other API elements due to the intent of the libraries�� designers to
provide certain functionality in API usages. Second, we expect that
with the advances in machine learning, deep learning~\cite{dnnbook}, and
statistical machine translation~\cite{smtbook}, which have effectively
captured the regularities and translate between natural languages, we
could leverage the regularity of API usages to generate API template
code from a given NL query.
%
In order to achieve our vision, we need to address fundamental
research problems:
% that have received little attention:
%from the software engineering (SE) community:

%graph-thing from NL to software

\noindent \textbf{Task 1. An empirical study to automatically build a
  large-scale parrallel corpus of text and code and understand the
  nature of text-code correspondence.}  First, we aim to build a
large-scale parallel corpus of texts and corresponding API usages.
The corpus will help us both in training any statistical model that we
will build for automatic generation of API code and in studying the
nature of text-code correspondence. The challenges are around how we
acquire large amounts of clean parallel NL/code data that is suitable
for training systems. The dataset must be ultra-scale, because the
statistical models are data-driven. They have the power to model
complex phenomena, but require substantial amounts of training data.
Second, the code snippets are usually missing information on the
correct libraries of the APIs. However, such API type information is
crucial in learning the mappings between text descriptions and API
code examples. Third, the mappings between texts and APIs that we mine
would help us understand whether a text phrase corresponds to a code
fragment, or AST's sub-tree, or a sub-graph in a program dependence or
control flow graph. Thus, it would help us to build better statistical
machine translation models to generate API code. 

To address those challenges, we are currently investigating the online
resources such as StackOverflow and GitHub. To avoid manual annotation
cost as in translation for NL texts, we utilize the modern
StackOverflow analysis tools. For example, we will use the ACE
tool~\cite{rigby-icse13} by Rigby and Robillard to automatically
extract code snippets and annotate them with the type information for
the API code elements. We will develop a method to derive type
information for incomplete code snippets on StackOverflow.


%API misuses and their relation to the security vulnerabilities. We
%have been conducting an ongoing empirical study on open-source
%projects and classifying them into different categories. We aim to
%develop an API-Misuse Classification (MUC) as a taxonomy for API
%misuses and a framework to assess the capabilities of misuse
%detectors. In a previous work, we develop MUbench, a data set of API
%misuses that we collected by reviewing over 1,200K reports from
%existing bug datasets and conduwcting a developer
%survey~\cite{ANNN+16}. MUBench provided us with the sample misuses
%needed to create a taxonomy. To cover the entire problem space of API
%misuses, we used Boa, our data mining infrastructure~\cite{boa-icse13}
%to further mine the misuses to this data set by looking at examples
%and claimed capabilities from API-misuse publications, and studies on
%API-usage directives.

%------------------------------

%Tien: come back here later
%OpenSSL is a software library for applications that secure communications over computer networks

\noindent {\bf Task 2. API Usage Code Synthesis: Models,
  Representations, and Methodologies.} The second major thrust of this
proposal will be to create models for code synthesis from NL text
descriptions. The key question is how we can create statistical code
synthesis models that are flexible, robust, and sensitive to the
context in which the API code is used. The challenges to achieve our
vision is the statistical translation models that are able to (1) make
use of large-scale amounts of data collected in the parallel corpus;
(2) be {\em programming language-oriented}, flexible and effectively
in representing API usage code (all existing work for code synthesis
have used directly existing machine translation models for NL that are
based on phrases, while code has structured and well-defined semantics
in term of data and control flows); (3) flexibly support several
languages for any application domains; (4) effectively represent
program properties in any domain.

To address those challenges, we plan to investigate and design a
context-sensitive, {\bf graph-based statistical translation} approach
that receives an English query on a programming~task and suggests an
API usage~template. We will choose graphs as a new representation and
will design graph-based translation models to fit with source code in
programming languages.
%
We will develop {\bf a graph synthesis algorithm for API
  usages}. First, the set of relevant API elements is derived from the
query by our novel {\bf contextual expansion algorithm} that maps the
words in the query into those elements.  Then, our model ensembles
those elements into a usage graph that represents an API usage with
data and control dependencies among its elements. For graph
synthesis, an API usage is represented by {\em a usage graph} that
models data and control dependencies among its API elements.
%
It starts from the pivotal API element in a graph and gradually adds
other nodes (and the corresponding inducing edges) via a beam search
strategy. It aims to cover the derived API elements and to maximize
both 1) the {\em relevance~to the words in the query} and 2) the
regularity of the newly built {\em API usage (sub)graph} (measured via
a graph-based language model GraLan~\cite{icse15} as its occurrence
likelihood within a large corpus of API usage graphs extracted from
code). With graphs, we support {\em partial orders among API
  elements} in API usages. Data and control dependencies help connect
relevant yet distant API elements in usages, and eliminate irrelevant
APIs even when they are~close.

%A benchmark infrastructure and dataset to evaluate API misuse
%detectors and a comparative study to discover their strengths and
%drawbacks. From the API misuses collected in the previous task, we
%extend MUBench to develop a benchmark infrastructure to empirically
%assess the peformance of existing API misuse detecting tools. In our
%preliminary findings, we reveal that existing detectors focus on very
%specific types of misuses and suffer from low precision and
%recall. More specifically, it identifies several major problems as
%follows: (1) Detectors fail to distinguish correct usages from
%misuses, due to not capturing sufficient code details in their
%representations; (2) Detectors ignore alternative usage patterns and
%semantically correct deviations, e.g., using a subtype versus a
%supertype, naively assuming that all deviations from frequent usages
%are misuses; (3) Detectors often fail to relate misuses with patterns,
%because they assume that a misuse misses only one or two facts and
%shares at least one fact with a pattern. However, some types of
%violations affect more than two facts from the underlying
%representation; (4) Detectors have poor ranking strategies. They rank
%many false positive findings higher than true positive ones.


%\begin{itemize}
%  \item Detectors fail to distinguish correct usages from misuses, due to not capturing sufficient code details in their representations.
%  \item Detectors ignore alternative usage patterns and semantically correct deviations, e.g., using a subtype versus a supertype, naively assuming that all deviations from frequent usages are misuses.
%  \item Detectors often fail to relate misuses with patterns, because they assume that a misuse misses only one or two facts and shares at least one fact with a pattern. However, some types of violations affect more than two facts from the underlying representation.
%  \item Detectors have poor ranking strategies. They rank many false positive findings higher than true positive ones.
%\end{itemize}
%

%We learn that evaluations mostly apply detectors in an intra-project
%setting; they learn correct usages solely from the project in which
%they detect misuses. A presented hypothesis is that detectors' low
%recall may be because individual projects do not contain enough usage
%examples for learning good~patterns.

%OLD
%From our study, we envision to support the identification
%(prediction/detection/prevention) and resolution (fixing) of similar
%security vulnerabilities on different systems based on the
%investigation of previously identified/resolved vulnerabilities on the
%systems that are similar in term of usage (i.e. software reuse).

\noindent {\bf Task 3. Integrating Code Synthesis into Developer
  Assistance Tools.}  Our goal in this task is to create developer
tools that allow them to enter a text description in NL, and produce an
API usage code that accurately implements the user��s request in the
description. With the capability of our models, we will create and
release tools for general consumption, solicit feedback from a wide
variety of developers, and examine how developers use our developer
assistance tools in an IDE or other enviroments, e.g., a Web-based API
that can be accessed programmatically by tool builders, plugins to
integrated development environments (e.g., Eclipse) that access the above
API, or a website where users can input their queries.

% The methodology and tool for identification/prevention and resolution
% of API-related vulnerabilities.} We will leverage the knowledge
%   learned from the empirical study on real-world API-related
%   vulnerabilities and strengths and weaknesses of existing API misuse
%   detectors to develop methods and tools for detecting and patching
%   the API-related vulnerabilities across different systems. As parts
%   of this task, we will develop the following: (1) the representation
%   for vulnerable code that sufficiently and effectively represent
%   security vulnerabilities as graph-based vulnerability models, with
%   nodes for software entities, and edges for their
%   relations/dependencies; (2) a mining algorithm for the patterns of
%   API usages that reflect the correct and incorrect usages; and (3)
%   an effective algorithm to detect API-related vulnerable code.
%
%Moreover, we focus on code-semantic-aware, frequent-subgraph-mining
%and graph-matching algorithms to identify patterns and violating
%instances. We will develop a systematically designed ranking strategy
%to improve precision.




%We propose to investigate a comprehensive and scientific methodology
%to achieve that ultimate goal. To do that, we first need to 1)
%understand the nature of similar software security vulnerabilities:
%whether they really recur because of the reuse? and how we could
%recognize such reuse? We also need to know the popularity and the
%severity of such vulnerabilities, as well as the characteristics of
%corresponding buggy software artifacts. With such knowledge, we will
%develop the techniques that 2) capture the knowledge of such
%vulnerabilities and their solutions, and then 3) leverage such
%knowledge in finding/detecting and patching the similar
%vulnerabilities across different systems.
%
%Our key philosophy for the approach is that: ``The \emph{reuse} of
%protocols, algorithms, procedures, modules, APIs, libraries,
%frameworks, or source code could cause \emph{similar vulnerabilities},
%i.e. when a bug or a security vulnerability occurs at one software
%entity, it likely occurs at the corresponding counter-parts of that
%entity''.
%
%Based on the philosophy, we propose the three following research tasks:
%
%\noindent \textbf{Task 1. An empirical study to understand the nature
%of similar security vulnerabilities and their relation to different
%levels of software reuse}.
%%the characteristics of reused software artifacts}.
%Among the software artifacts related to vulnerabilities, we will
%emphasize on security reports and source code. We plan to read and
%analyze the security reports, and corresponding source code files and
%patches to answer the following questions: How popular and severe are
%the similar vulnerabilities?  Do they recur all due to the reuse
%practice? What are the levels of recurrence/similarity/reuse:
%specification, protocol, algorithm, design, code? What are the
%features of the reports and code fragments corresponding to the
%vulnerabilities that could help us characterize them and recognize the
%similar ones, and from there, develop automated methods to prevent
%them?
%
%%Our main hypothesis in this empirical study is that: reuse (protocols,
%%algorithms, modules, libraries, source code files) causes similar security vulnerability, i.e. .
%%similar code, similar code has similar bug, thus, causing similar
%%vulnerabilities.
%
%\noindent \textbf{Task 2. Scientific foundation for the representation
%and similarity measurement for software vulnerabilities}. We plan to
%use the knowledge gained from Task 1 to investigate the theoretical
%foundation for the \emph{representation} and \emph{similarity
%measurements} of security vulnerabilities and relevant code
%entities. Based on our philosophy of reuse, the desired representation
%and similarity measurements for vulnerabilities will mainly rely on
%the usage relation(s) among the entities in software systems. The
%usage relation could be at different levels: software components,
%modules, libraries, or code entities (classes, methods, functions),
%etc.  We will use it to represent the security vulnerabilities as
%graph-based vulnerability models, with nodes for software entities,
%and edges for their relations/dependencies. Then, we will develop
%techniques to build such representation models from vulnerability
%reports written in a  standard form and from corresponding code
%entities and their fixes. We will develop graph-based techniques
%to measure the similarity of such models.
%
%%The ultimate goal of this task is to identify if a
%%newly reported vulnerability is similar to an existing one. Normally,
%%a newly reported vulnerability is expressed in some standard form
%%(e.g. CVE format~\cite{cve}). Therefore, we will develop a technique
%%to extract important information from the  and convert them into our representation
%%so that we could use our similarity measurement among the
%%vulnerabilities.
%
%\noindent \textbf{Task 3. The methodology for
%identification/prevention and resolution of similar
%vulnerabilities}. We will develop a novel methodology that, based on
%the knowledge of the prior known vulnerability (e.g. its
%characteristics, the location of corresponding buggy code, and how it
%was fixed), \emph{identifies} the similar vulnerability in other
%systems and \emph{suggest} the resolution for that vulnerability. As
%parts of this task, we will develop the following {\em algorithms}:
%
%1) to map software entities between security reports and from a report
%   to corresponding source code fragments/modules/components. This
%   algorithm will be mainly based on textual analysis and information
%   retrieval techniques for locating the concepts between different
%   artifacts.
%
%2) to determine the modules/source file locations in the new system
%that correspond to the buggy modules/locations in the known
%vulnerability. The algorithm is mainly based on finding similar code.
%
%%For example, if they reuse the same algorithm, we will find the code in the new system that implements the same algorithm as the previously
%%buggy system.
%
%3) to derive the patching/fixing to the new system from the prior
%fix. Since the systems having similar vulnerabilities might have differences at several levels, we will develop different
%resolution techniques, ranging from recommendations, suggestions, to
%automatic patches.
%We will show suggested fixes, e.g., the fixes in source files or updates in API modules.

%For the algorithm 1), we will rely on the reuse between two
%systems. For example, if they reuse the same algorithm, we will find
%the code in the new system that implements the same algorithm as the
%previously buggy system. For the algorithm 2), we will use the
%information retrieval techniques for locating the concepts that a
%report mentions about. For example, a vulnerability report refers to
%the source files that handle the connection to a Web server.
%algorithm 2 will analyze the textual description in the report and
%automatically locate those source files. For the algorithm 3), we will
%limit to the recommendation and suggestion only. We will show
%suggested fixes, for example, the fixes in source files or updates in
%API modules.

%Based on the techniques developed in task 2, and the knowledge of
%known vulnerabilities (code/document/report), we could be able to
%identify locations of potential similar vulnerabilities and provide
%patching/fixing recommendation. That is, when a vulnerability is
%discovered on a system, we will compare that system (or corresponding
%module(s)) to other systems, to determine whether they have similar
%reuse related to that vulnerability.

%The mapping between the vulnerability report and the source code, as
%well as the patches of the first system, could be used to recommend
%the potential location for patching and the corresponding patches for
%the other systems.

%Similarity and mapping between those code modules/reports is derived
%based on each level of reuse: code, API, algorithm, etc.
